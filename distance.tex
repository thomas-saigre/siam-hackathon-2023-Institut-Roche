\documentclass{article}

\usepackage[english]{babel}
\usepackage[latin1]{inputenc}
\usepackage[T1]{fontenc}
\usepackage{amsmath}
\usepackage{amssymb}
\usepackage{bm}
\usepackage{listings}
\lstset{language=matlab, breaklines=true}
\usepackage{graphicx}
\usepackage{placeins}

\author{Alberto Bocchinfuso}
\title{Siam hackathon}

\begin{document}
	\maketitle

\bigskip
{\Large {\bf Distance function}}
\bigskip

Given
\begin{equation}
	A = \begin{bmatrix}
		a_1^\top \\
		\vdots \\
		a_m^\top
	\end{bmatrix}
\end{equation}
find a distance function
\begin{equation}
	d(a_i, \bar{a}, \Sigma, a^{M})
\end{equation}
that weights how far $a_i$ is form $\bar{a}$, mean value of $A$, taking into consideration the covariance matrix $\Sigma$ and the vector of maximum differences
\begin{equation}
	a^{M}_j = \max_{i} \, \vert a_{i,j} - \bar{a}_j \vert \, .
\end{equation}

Possible distances are based on energy norms with respect to the covariance
\begin{equation}
	d_1(a, \bar{a})^2 = (a - \bar{a})^\top \Sigma^{-1} (a - \bar{a})
\end{equation}
or with respect to the square of the biggest difference:
\begin{equation}
	d_2(a, \bar{a})^2 = (a - \bar{a})^\top M^{-1} (a - \bar{a})
\end{equation}
with M:
\begin{equation}
\begin{bmatrix}
	\left[ a_1^M \right]^2	&0	&0	&\dots 	&0 \\
	0	&\left[ a_2^M \right]^2	&0	&\dots	&0 \\
	\vdots	&0	&\ddots	&	&0 \\
	\vdots	&\vdots	&	&\ddots	&0 \\
	0	&0	&0	&\dots	&\left[ a_N^M \right]^2
\end{bmatrix} \, .
\end{equation}

The last possible distance is a convex combination of $d_1, d_2$:
\begin{equation}
	d_3(a, \bar{a})^2 = \theta d_1(a, \bar{a}) + (1 - \theta) d_2(a, \bar{a}) \, , \; \theta \in (0, 1)
\end{equation}

The metric to measure how good the privatization is, we may use:
\begin{equation}
	M(A, A^1) = \sum_{i=1}^{m} d(a, \bar{a}) (1 - P(j| a_j^1) )
\end{equation}
where $d$ is one of the distances above, and $P(j | a_i^1)$ is a measure of the probability to recognize the that $i$ comes from the $j$-th row of $A$, assuming that $j$ is really the correct row. 

\textbf{What we do}:
\begin{itemize}
	\item If the matrix is the same, the measure $M(A, A) = 0$
	\item If the matrix is the same and the masking is shuffling the rows, the measure $M(A, LA) = 0$ ($L$ is a permutation matrix)
	\item We weight more the data that are further from the mean value.
	\item The distance is computed from the original matrix $A$, because it is possible that the privatization moves the outliers very close to the mean, so we still want to consider the fact that they were outliers in the original collected data.
	\item If there is a moderate probability to retrieve $j$ from $a_i^1$ for data near the average $\bar{a}$, we don't weight it too much. A particle which is similar to $\bar{a}$ is probably confusing with many other particles.
\end{itemize}
\textbf{What we expect}:
\begin{itemize}
	\item $d_2$ works well in the presence of many outliers because it is not weighting them too much, since it considers the maximum difference and not just the standard deviation, which may lead to weight a lot the single outlier.
\end{itemize}

This metric shouldn't give too much advantage of the fact that a masking method masks the outliers very well since considers all the points and weights them both according to the distance from the mean (how peculiar that particle is in the dataset.), but also according to the probability of being recognized.

We expect that the values that are close to the mean are the most present, so if the algorithm doesn't mask well those data, we will weight this fact: we sum many small numbers ($1-P(j| a_j^1)$) so we keep the metric close to $0$), too.

$M(A,A^1)$ close to $0 \implies $ the matrices are very similar, in the sense that an attacker can retrieve the specific particles. $M(A,A^1)$ big $\implies$ the masking matrix $A^1$ doesn't allow an attacker to retrieve specific particles, in general. It is still possible to recognize some specific people.

In practice, there should be a trade off between data quality and $M(A,A^1)$.


\end{document}